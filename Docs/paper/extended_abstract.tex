\chapter*{Extended Abstract}

\begin{center}
	\begingroup
	\renewcommand*{\arraystretch}{1}
	\rowcolors{2}{white}{white}
	{\makeatletter	
		\begin{tabular}{p{3.2cm}p{9.6cm}}
			Thema: & \thema \\
			& \\
			Teammitglieder: & \verfasserA, \verfasserB \\
			& \\
			Betreuer: & \hoschschule \newline \institut \newline \prueferA, \prueferB \\
			& \\
		\end{tabular}
		
		\makeatother}
	\endgroup
\end{center}

\bigskip

We try to predict precipitation for a range of 35 minutes in an area around Constance.
Therefore we are using machine learning techniques and train a UNet on radar data images. 
Here we present the result of precipitation prediction as well with regression as with classification. 
Both approaches provide good results.


\printbibliography[title={Referenzen}, heading=subbibliography]

